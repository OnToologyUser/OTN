    \item \emph{título oficial} \label{def_titulo_documento} Documento expedido
        en nombre del Rey por el Rector de una universidad \cite[artículo
        35]{leyUniversidades} que acredita una serie de conocimientos
        científicos, técnicos o artísticos \cite[artículo
        33]{leyUniversidades}.

    \item \emph{plan de estudios}. Programa en el que se detallan las
        asignaturas que hay que cursar para obtener un título, así como los
        medios, tanto humanos como materiales, que se van a emplear para
        llevara buen término el programa.

    \item \emph{universidad}. Institución de educación superior dedicada a la
        investigación, la docencia y el estudio \cite[articulo 1, apartado 
        1]{leyUniversidades}.

    \item \emph{universidad pública}. Universidad cuya titularidad ostenta el Estado o
        una Comunidad Autónoma \cite[artículo 3]{rdUniversidadesPrivadas}.

    \item \emph{universidad privada}. Universidad cuya titularidad ostenta una persona
        física o jurídica de carácter privado\cite[artículo
        3]{rdUniversidadesPrivadas}.

    \item \emph{centro 1}. División académica de una universidad, en la que se agrupan
                los estudios de disciplinas similares (adaptado de la
                RAE\footnote{\url{http://dle.rae.es/?id=HTxyZDZ}}). Un centro
                está encargado de la organización de las enseñanzas y de los
                procesos académicos, administrativos y de gestión conducentes a
                la obtención de títulos universitarios \cite[artículo 8]{leyUniversidades}.

     \item \emph{centro 1} Local o conjunto de locales en que funciona un centro según la acepción \ref{centro_div} (adaptado de la
                RAE\footnote{\url{http://dle.rae.es/?id=HTxyZDZ}}).
        \end{enumerate}

    \item \emph{departamento}. unidad de docencia e investigación encargada de
        coordinar las enseñanzas de uno o varios ámbitos del conocimiento en uno o varios centros,
        de acuerdo con la programación docente de la universidad, de apoyar las actividades e
        iniciativas docentes e investigadoras del profesorado, y de ejercer aquellas otras funciones
        que sean determinadas por los estatutos\cite[artículo 9]{leyUniversidades}.

    \item \emph{curso académico}. Intervalo de tiempo anual en el que se desarrolla la actividad académica. 

    \item \emph{enseñanza presencial}. Aquélla que requiere que el estudiante
        asista de forma regular y continuada durante todos los cursos a
        actividades formativas regladas en el centro de impartición del título\cite[página 10]{guiaMadridMasD}.

    \item \emph{enseñanza semi-presencial}. Aquélla en la que la planificación de las
        actividades formativas previstas en el Plan de Estudios combina la
        presencia física del estudiante en el centro de impartición del título con un
        mayor trabajo autónomo del estudiante al propio de la enseñanza
        presencial\cite[página 10]{guiaMadridMasD}.

    \item \emph{enseñanza a distancia}. Aquélla en que la gran mayoría de las actividades
        formativas previstas en el Plan de Estudios no requieren la presencia física
        del estudiante en el centro de impartición del título\cite[página 10]{guiaMadridMasD}.

    \item \emph{crédito ECTS}. El crédito europeo es la unidad de medida del
        haber académico que representa la cantidad de trabajo del estudiante
        para cumplir los objetivos del programa de estudios y que se obtiene
        por la superación de cada una de las materias que integran los planes
        de estudios de las diversas enseñanzas conducentes a la obtención de
        títulos universitarios de carácter oficial y validez en todo el
        territorio nacional. En esta unidad de medida se integran las
        enseñanzas teóricas y prácticas, así como otras actividades académicas
        dirigidas, con inclusión de las horas de estudio y de trabajo que el
        estudiante debe realizar para alcanzar los objetivos formativos propios
        de cada una de las materias del correspondiente plan de
        estudios\cite[artículo 3]{rdECTS}. El número mínimo de horas, por
        crédito, será de 25, y el número máximo, de 30\cite[artículo 4, apartado 
        5]{rdECTS}.

    \item \emph{estudiante}. Toda  persona  que  curse  enseñanzas  oficiales
        en alguno de los tres ciclos universitarios (Grado, Máster y Doctorado
        \cite[artículo 8]{rd1393}), enseñanzas de formación continua u otros
        estudios ofrecidos por las universidades\cite[artículo
        1, apartado 3]{estatutoEstudiante}.

    \item \emph{estudiante a tiempo completo}. Estudiante que dedica un mínimo
        de 36 horas y un máximo de 40 a la semana a sus
        universitarios\cite[artículo 4, apartado 4]{rdECTS}.

    \item \emph{estudiante a tiempo parcial}. Estudiante que requiere una
        trayectoria de aprendizaje flexible \cite[artículo 7, apartado 
        2]{estatutoEstudiante}, y que, por tanto, no llega a una dedicación de
        36 horas semanales a sus estudios universitarios. 

    \item \emph{normativa de permanencia}. Conjunto de normas que establecen
        bajo qué condiciones un estudiante puede permanecer en la universidad.

    \item \label{idioma} \emph{idioma}. Lengua de un pueblo o nación, o común a
        varios\footnote{\url{http://dle.rae.es/?id=KuMp7nw} acepción 1}.

    \item \emph{competencia}. Véase la sección \ref{competencias}.
        En el ámbito universitario, las competencias deben ser
        evaluables y coherentes \cite[página 19]{guiaAneca}.

    \item \emph{competencia básica}. Véase la sección \ref{competencias}.

    \item \emph{competencia específica}. Véase la sección \ref{competencias}.

    \item \emph{competencia transversal}. Véase la sección \ref{competencias}.

    \item \emph{profesión}. Empleo, facultad u oficio que alguien ejerce y por
        el que percibe una
        retribución\footnote{\url{http://dle.rae.es/?id=UHx86MW} acepción 2}.

    \item \emph{profesión regulada por exigencia de título universitario}:
        según \cite[artículo 4, apartado c]{rdECTS}, aquella profesión para cuyo
        acceso se exija estar en posesión de un título universitario oficial
        cuyo diseño y directrices respondan a lo dispuesto en los artículos
        12.9 y 15.4 del Real Decreto 1393/2007\cite{rd1393}.

    \item \emph{descripción en lenguaje natural}. Descripción utilizando un idioma humano (véase \ref{idioma}). 

    \item \emph{número de plazas}. Número de nuevos estudiantes que pueden
        admitirse a lo largo de un curso académico en una titulación.

    \item \emph{plazo de preinscripción}. Intervalo de tiempo en el que se
        puede solicitar el ingreso en una tituación.

    \item \emph{calendario de implantación del título}. Secuencia temporal en la
        que se van implantando los diferentes cursos de la
        titulación\cite[página 62]{guiaAneca}.

    \item \emph{rama de conocimiento}. Área del saber que, según el RD 1393
        \cite[artículo 12, apartado 4]{rd1393}, puede ser una de las siguientes:
        Artes y Humanidades, Ciencias, Ciencias de la salud, Ciencias Sociales
        y Jurídicas, e Ingeniería y Arquitectura. Todo título debe estar
        adscrito a, al menos, una rama de conocimiento.

    \item \emph{materia}. Unidad académica que incluye una o varias asignaturas que
        pueden concebirse de manera integrada, de tal forma que constituyen
        unidades coherentes desde el punto de vista disciplinar \cite[página
        33]{guiaAneca}.

    \item \emph{módulo}. Unidad académica que incluye una o varias materias que
        constituyen una unidad organizativa dentro de un plan de estudios
        \cite[página 33]{guiaAneca}.

    \item \emph{asignatura}. Unidad académica que incluye una serie de
        contenidos coherentes desde el punto de vista disciplinar de la que es
        evaluado el estudiante y se le asigna una calificación.

    \item \emph{carácter de materia/asignatura}. Clasificación de la materia/asignatura en
        básica, obligatoria, optativa, seminario, prácticas externas,
        trabajo dirigido, trabajo fin de grado, trabajo de fin de máster, o
        mixta.\cite[artículo 12, apartado 2]{rd1393} \cite[sección
        D.5]{guiaMadridMasD}

    \item \emph{materia/asignatura básica}. Materia/asignatura en que el estudiante adquiere las
        competencias fundamentales de la rama de conocimiento \cite[artículo
        12, apartado 2]{rd1393}. El RD 1393 \cite[anexo II]{rd1393} proporciona
        un catálogo de materias básicas por ramas de conocimiento.

        El plan de estudios de cualquier titulación deberá contener un mínimo de
        60 créditos de formación básica, de los que, al menos, 36
        estarán vinculados a algunas de las materias del listado anterior
        para la rama de conociento a la que esté adscrito el título\cite[artículo 12,
        apartado 5]{rd1393}.

        Estas materias deberán concretarse en asignaturas con un mínimo
        de 6 créditos cada una y serán ofertadas en la primera
        mitad del plan de estudios.

        Los créditos restantes hasta 60, en su caso, deberán estar configurados por
        materias básicas de la misma u otras ramas de conocimiento de las incluidas en
        el listado anterior, o por otras materias siempre que se justifique su carácter
        básico para la formación inicial del estudiante o su carácter transversal.

    \item \emph{materia/asignatura obligatoria}. Aquélla no básica que debe ser cursada
        necesariamente para obtener el título\cite[sección
        D.5]{guiaMadridMasD}\cite[artículo 12, apartado 2]{rd1393}.

    \item \emph{materia/asignatura optativa}. Aquélla que elige el estudiante entre
        otras posibles para cursarla\cite[sección
        D.5]{guiaMadridMasD}\cite[artículo 12, apartado 2]{rd1393}.

    \item \emph{materia mixta}. Materia que engloba asignaturas de diferente
        carácter\cite[sección D.5]{guiaMadridMasD}\cite[artículo 12, apartado 
        2]{rd1393}. 

    \item \emph{itinerario formativo}. Mención (para el caso de un grado),
        especialidad (para el caso de un máster) \cite[sección
        D.4]{guiaMadridMasD}\cite[artículo 9, apartado 3]{rd1393}.

    \item \emph{trabajo fin de grado}. Trabajo con el que concluye un título de
        grado\cite[artículo 12, apartado 3]{rd1393}. El trabajo de fin de Grado
        tendrá entre 6 y 30 créditos, deberá realizarse en la fase final del
        plan de estudios y estar orientado a la evaluación de competencias
        asociadas al título.\cite[artículo 12, apartado 7]{rd1393}

    \item \emph{trabajo fin de máster}. Trabajo con el que concluye un título
        de máster\cite[artículo 15, apartado 3]{rd1393}. Tendrá entre 6 y 30 créditos.

    \item \emph{guía docente}. Especificación de una asignatura que incluye su
        número de créditos ECTS, el programa a desarrollar, la metodología de
        aprendizaje, los sistemas de evaluación, etc. \cite[sección
        D.5]{guiaMadridMasD}.

    \item \emph{profesor}. Persona que se dedica a la docencia en la
        universidad.

    \item \emph{profesor ayudante}. Profesor en periodo de formación, con
        contrato de carácter temporal y dedicación a tiempo completo, que ha
        sido admitido, o está en condiciones de ser admitido, en estudios de
        doctorado\cite[artículo 49]{leyUniversidades}. 

    \item \emph{profesor ayudante doctor}. Profesor doctor, con contrato de
        carácter temporal y dedicación a tiempo completo, que ha obtenido una
        acreditación nacional o autonómica, y que desarrolla tareas docentes y de
        investigación\cite[artículo 50]{leyUniversidades}.  

    \item \emph{profesor titular de universidad}. Profesor funcionario doctor con plena
        capacidad tanto para la docencia como para la
        investigación. Para acceder a esta posición es necesario haber obtenido
        una acreditación nacional\cite{leyUniversidades}.

    \item \emph{catedrático de universidad}. Profesor funcionario doctor con plena
        capacidad tanto para la docencia como para la investigación que tiene
        la categoría docente más alta. Para acceder a esta posición es
        necesario haber obtenido una acreditación nacional\cite{leyUniversidades}.

    \item \emph{profesor contratado doctor}. Profesor doctor con contrato
        laboral indefinido y dedicación a tiempo completo que ha obtenido una
        acreditación nacional o autonómica, que desarrolla, con plena capacidad
        docente e investigadora, tareas docentes y de investigación (o
        prioritariamente de investigación)\cite[artículo 52]{leyUniversidades}.

    \item \emph{profesor asociado}. Profesor, con contrado temporal a tiempo
        parcial, especialista de reconocida competencia que acredita ejercer su
        actividad profesional fuera del ámbito académico universitario que
        desarrolla tareas docentes a través de las que aporta sus conocimientos
        y experiencia profesionales a la universidad\cite[artículo 53]{leyUniversidades}.

    \item \emph{profesor visitante}. Profesor o investigador, con contrato
        temporal a tiempo parcial o completo, de reconocido prestigio de otra
        universidad o centro de investigación, que desarrolla tareas docentes o
        de investigación a través de las que aporta conocimientos y experiencia
        docente e investigadora\cite[artículo 54]{leyUniversidades}.

    \item \emph{profesor emérito} Profesor jubilado que ha que ha prestado
        servicios destacados a la universidad\cite[artículo 54bis]{leyUniversidades}.

    \item \emph{doctor}. Persona que ha defendido con éxito una tesis doctoral,
        consistente en un trabajo original de investigación y que, por tanto,
        posee el título de doctor.

    \item \emph{aula}. Sala donde se imparten clases en los centros
        docentes\footnote{\url{http://dle.rae.es/?id=4OCO4gi} acepción 1}.

    \item \emph{aula informática}. Aula con ordenadores para que puedan
        trabajar con ellos los alumnos.

    \item \emph{recurso bibliográfico}. Libro, revista o cualquier otra obra
        científica o literaria que puede estar impresa o en otro soporte. 

    \item \emph{biblioteca}. Órgano cuya finalidad consiste en la adquisición,
        conservación, estudio y exposición de libros y
        documentos\footnote{Inspirado en \url{http://dle.rae.es/?id=5SGETnQ}
        acepción 1}.

    \item \emph{sala de estudio}. Sala debidamente acondicionada para que los
        alumnos puedan estudiar. 

\bibliography{../bibliografia/bibTex}{}
\bibliographystyle{plain}
